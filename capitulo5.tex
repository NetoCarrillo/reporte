\chapter{Conclusiones}\label{cap5}

A lo largo de este trabajo se ha descrito el proceso de desarrollo del Sistema AutoSA, se ha comprobado que la implementación satisface los requerimientos funcionales y no funcionales mencionados en el Capítulo \ref{cap2}, el cual expone las bases para mostrar el cumplimiento de los objetivos del proyecto AutoSA.\\
El objetivo principal del proyecto AutoSA es automatizar la interacción de los operadores de la farmacéutica con el \textit{Sistema de Abastecimiento} para contestar y verificar órdenes de reposición del \textit{Instituto de Salud}. En las primeras semanas desde la liberación del sistema AutoSA se respondió exitosamente la totalidad de las órdenes de reposición, que en promedio fueron 400 órdenes por día, en tanto que el tiempo de atención promedio por lote fue de hora y media. Cabe resaltar que hasta la fecha (octubre de 2019) no se han reportado defectos graves o críticos, tampoco errores en los datos relacionados con la respuesta a las órdenes de reposición. Por lo anterior se puede decir que el proyecto AutoSA cumple con el objetivo para el cual fue propuesto. La consultora dueña del desarrollo del sistema AutoSA, posterior a la liberación del mismo, lo ha implantado en otras compañías farmacéuticas teniendo resultados similares a los obtenidos en la primera compañía.\\
El uso del sistema AutoSA trajo con sigo los beneficios esperados:
\begin{enumerate}
	\item Reducción de tiempo en cuanto a la respuesta de órdenes de reposición. Previo al uso del sistema AutoSA, a los operadores de la farmacéutica les tomaba 24 horas hombre al día contestar 400 órdenes de reposición; con el sistema AutoSA el tiempo de respuesta bajó a 1.5 horas. Este hecho ocasionó que todo el proceso de la compañía farmacéutica para entregar el medicamento desde que se publicaron las órdenes de reposición en el \textit{Sistema de Abastecimiento} bajara de uno a dos días.
	\item Reducción de horas extras en las jornadas laborales de los operadores de la compañía farmacéutica. Debido al ahorro de tiempo en la respuesta a las órdenes de reposición, los operadores pueden realizar sus tareas diarias dentro de la jornada laboral de 8 horas.
	\item Reducción del error humano en relación con la manipulación de la información. Hasta la fecha no se han reportado errores o inconsistencias en los datos de las respuestas a las órdenes de reposición.
	\item Consistencia en los datos respecto a la generación de reportes estadísticos sobre las órdenes de reposición procesadas. Del punto anterior se conoce que no existen inconsistencias en los datos de las órdenes de reposición almacenados en la base de datos, por lo que los reportes estadísticos muestran datos consistentes y verdaderos.
	\item Ahorro de recursos en la entrega de medicamentos no solicitados, puesto que la farmacéutica es capaz de revisar con mayor frecuencia la cancelación de órdenes por parte del \textit{Instituto de Salud}. Esto le ha dado mayor oportunidad para detener el envío de medicamento cuya solicitud ha sido cancelada, ahorrando así recursos económicos y materiales. La información exacta de esta reducción no ha sido compartido por la compañía farmacéutica.
\end{enumerate}
Este reporte ha mostrado el desarrollo del proyecto AutoSA dividido en:
\begin{itemize}
	\item Presentación del problema y propuesta de solución. 
	\item Análisis de requerimientos y creación de casos de uso.
	\item Diseño de arquitectura y componentes.
	\item Implementación de los componentes.
\end{itemize}
Durante todas estas etapas se trabajó en constante comunicación con los operadores de la farmacéutica. En una primera etapa para copiar exactamente la interacción con el \textit{Sistema de Abastecimiento} del \textit{Instituto de Salud}. Posteriormente para la ejecución de pruebas de las rutinas automatizadas, pues al no tener un ambiente de pruebas en el \textit{Sistema de Abastecimiento}, estás fueron realizadas directamente en el ambiente de producción. Por lo que la supervisión de los operadores fue necesaria para asegurar que toda orden de reposición fuera atendida por parte de la farmacéutica y además garantizar que los datos de las órdenes fueran almacenados correctamente, esto es, que tanto la generación de los acuses de envió fuera correcta, como el contenido del reporte de órdenes atendidas.\\
El resultado del proyecto AutoSA es un sistema robusto que, si bien tiene puntos de mejora, al día de hoy no ha presentado fallos graves o críticos, ha cubierto las necesidades de la compañía farmacéutica superando el ahorro de tiempo previsto, ha reducido costos por errores humanos y costos por errores en relación con la logística, además de evitar a los operadores de la farmacéutica jornadas laborales de 10 horas.\\
El proyecto AutoSA ha mostrado ser un caso de éxito en la industria farmacéutica, ya que este mismo sistema ha sido requerido por otras compañías farmacéuticas y hasta el día de hoy sigue siendo  utilizando diariamente.
\paragraph{Trabajo futuro\\}
Se sugieren las siguientes tareas para mejorar y ampliar las funcionalidades del sistema AutoSA:
\begin{itemize}
	\item Actualizar las bibliotecas, los marcos de trabajo y los ambientes de ejecución, ya que han pasado más de tres años desde la liberación del sistema.
	\item Ejecutar en paralelo y acondicionar el sistema AutoSA para automatizar la respuesta de órdenes de reposición en varias instancias de la herramienta \textit{Sahi}. Esto con el fin de reducir aún más el tiempo de respuesta a las órdenes de reposición.
	\item Extender el alcance del sistema para interactuar con el sistema de la farmacéutica que maneja el inventario de la bodega de medicamentos. Lo cual quiere decir que al terminar la atención de las órdenes de reposición, el sistema AutoSA enviaría el reporte de los medicamentos solicitados directamente a la bodega, agilizando así el proceso de entrega de medicamentos a los centros de salud. 
	\item Programación de la ejecución de las rutinas de automatización. De esta forma no se necesitaría un operador que iniciara manualmente las rutinas.
\end{itemize}
En el desarrollo de este proyecto y en toda mi carrera profesional he aplicado conocimientos que adquirí en la Facultad de Ciencias sobre programación, lógica, bases de datos, redes de computadoras, ingeniería de software, análisis y diseño de algoritmos, teoría de códigos y álgebra lineal, por mencionar algunos. De igual manera, también he utilizado las habilidades y costumbres que adquirí como alumno de la Licenciatura en Ciencias de la Computación, tales como buscar información y aprender de forma autodidacta nuevas tecnologías, así como pensar soluciones alternativas. Gracias a lo anterior, he podido mantenerme actualizado en tendencias para el desarrollo de software, lo que me ha permitido ser un profesional competente en los equipos de trabajo de los cuales he formado parte.\\
