\subsection{Portal Web}
La implementación del portal Web se realizó bajo la arquitectura orientada a servicios (ver Apéndice \ref{sec-soa}), lo cual implica una división del portal Web de la siguiente forma:
\begin{itemize}
	\item [Backend]: la parte del portal ofrece los servicios Web, con tiene los servicios de autenticación, administración de órdenes de reposición y generación de reportes.
	\item [Frontend]: la parte del portal que consume los servicios Web y muestra una interfaz gráfica al usuario, contiene las páginas de HTML y rutinas para consumir los servicios del Backend.
\end{itemize}

\subsubsection{Implementación del Backend}
La implementación del Backend está implementada utilizando el marco de trabajo Spring y sus bibliotecas:
\begin{itemize}
 	\item Spring boot: provee un ambiente de desarrollo y ejecución para aplicaciones basadas en el marco de trabajo Spring (ver sección \ref{sec-spring-boot}). 
 	\item Spring security: biblioteca para los servicios de autenticación y autorización de usuarios (ver sección \ref{sec-spring-security}).
\end{itemize}

\paragraph{Implementación de los servicios de seguridad\\}
Los servicios para autenticar y validar usuarios se implementaron siguiendo la especificación OAtuh 2.0 (ver sección \ref{sec-oauth}), el marco de trabajo Spring con sus bibliotecas Spring boot y Spring security.\\
La implementación de la seguridad utilizando Spring Security se divide en tres tareas:
\begin{enumerate}
	\item Habilitar filtros de seguridad, en el Código \ref{lst:enable-oauth} se muestra la configuración de necesaria:
	\begin{enumerate}
		\item Línea 2: habilita el uso de la especificación OAuth2.
		\item Línea 7: configuración principal de Spring security para no interponerse con los filtros de OAuth.
		\item Línea 12: creación del filtro de OAuth.
	\end{enumerate}
\begin{lstlisting}[language=Java, caption={Clase para habilitar los filtros de seguridad.}, captionpos=b, label={lst:enable-oauth}]
@Configuration
@EnableOAuth2Client
@Order(SecurityProperties.ACCESS_OVERRIDE_ORDER)
public class WebSecurityConfig extends WebSecurityConfigurerAdapter{

	@Override
	protected void configure(HttpSecurity http) throws Exception{
		http.antMatcher("/**").authorizeRequests().anyRequest().authenticated().and().logout().logoutSuccessUrl("/").permitAll().and().sessionManagement().sessionCreationPolicy(SessionCreationPolicy.STATELESS).and().csrf().disable();
		http.authenticationProvider(authenticationProvider);
	}
	
	@Bean
	public FilterRegistrationBean oauth2ClientFilterRegistration(
			OAuth2ClientContextFilter filter){
		FilterRegistrationBean registration = new FilterRegistrationBean();
		registration.setFilter(filter);
		registration.setOrder(-100);
		return registration;
	}
}	
\end{lstlisting}

	\item Habilitar servicios de autorización: se refiere a dar un token de acceso (como lo especifica OAuth 2.0), en el código \ref{lst:enable-auth-server} se muestra la declaración de una clase para la configuración de un servidor de autorización:

\begin{lstlisting}[language=Java, caption={Clase de autenticación de usuarios.}, captionpos=b, label={lst:enable-auth-server}]
@Configuration
@EnableAuthorizationServer
public class AuthorizationServerConfig extends AuthorizationServerConfigurerAdapter{
}
\end{lstlisting}

	En el Código \ref{lst:user-auth} se muestra la configuración necesaria para autenticar usuarios:
	\begin{enumerate}
		\item Línea 2: inyección del bean que realiza la autenticación de usuarios.
		\item Línea 4: creación del bean encargado de almacenar los tokens de los usuarios.
		\item Línea 10: configuración del punto de entrada para autenticación.
		\item Línea 11: se establece la referencia al bean que autentica usuarios.
		\item Línea 12: se establece la referencia al bean que administra los tokens de los usuarios.
	\end{enumerate}

\begin{lstlisting}[language=Java, caption={Configuración de autenticación de usuarios.}, captionpos=b, label={lst:user-auth}]
@Autowired
private AuthenticationManager authenticationManager;

@Bean
public TokenStore tokenStore(){
	TokenStore tokenStore = new InMemoryTokenStore();
	return tokenStore;
}

@Override
public void configure(AuthorizationServerEndpointsConfigurer endpoints) throws Exception{
	endpoints.authenticationManager(authenticationManager);
	endpoints.tokenStore(tokenStore());
}
\end{lstlisting}

	En el Código \ref{lst:client-auth} se muestra la configuración para la autenticación del cliente (Frontend):

	\begin{enumerate}
		\item Líneas 1 a 4: lectura de las credencia les del cliente.
		\item Línea 10: configuración del protocolo para la autenticación del cliente. 
	\end{enumerate}

\begin{lstlisting}[language=Java, caption={Clase de autenticación de cliente.}, captionpos=b, label={lst:client-auth}]
@Value("${oauth.server.client.id}")
private String clientId;
@Value("${oauth.server.client.secret}")
private String clientSecret;

@Autowired
private EncodedClientDetailsService ecds;

@Override
public void configure(ClientDetailsServiceConfigurer clients) throws Exception{
	BaseClientDetails details = new BaseClientDetails();
	details.setClientId(clientId);
	details.setClientSecret(clientSecret);
	details.setAuthorizedGrantTypes(Arrays.asList("password"));
	ecds.addClientDetails(details);
	clients.withClientDetails(ecds);
}
\end{lstlisting}


	\item Habilitar acceso a recursos: estos recursos pueden ser los elementos estáticos que muestra el explorador de Internet, es decir, rutinas de javascript, páginas HTML, hojas de estilo e imágenes; o el consumo de servicios Web en el Código \ref{lst:enable-resource-server} se muestra la configuración del servidor de recursos:

	\begin{enumerate}
		\item Línea 2: habilitar el servidor de recursos.
		\item Línea 5: configuración del servidor de recursos.
		\item Línea 7: enunciación de las URLs públicas.
		\item Línea 8: enunciación de las URLs que requieren de un usuarios autorizado. 
	\end{enumerate}

\begin{lstlisting}[language=Java, caption={Clase de configuración de servidor de recursos.}, captionpos=b, label={lst:enable-resource-server}]
@Configuration
@EnableResourceServer
public class ResourceServerConfig extends ResourceServerConfigurerAdapter{
	@Override
	public void configure(HttpSecurity http) throws Exception{
		http.authorizeRequests()
			.antMatchers(PUBLIC_URLS).permitAll()
			.anyRequest().authenticated();
	}
}
\end{lstlisting}
\end{enumerate}

\paragraph{Implementación de los servicios Web de administración\\}
Los servicios web de administración fueron divididos en dos controles REST de Spring, 
\begin{enumerate}
	\item \textbf{DataController}: expone servicios referentes a la gestión de órdenes de reposición, en el Código \ref{lst:data-controller} se muestra la estructura principal de esta clase:
	\begin{enumerate}
		\item Línea 1: indicación para crear un un bean que expone servicios REST.
		\item Línea 3: inyección del bean de MyBatis para administrar órdenes de reposición.
	\end{enumerate}

\begin{lstlisting}[language=Java, caption={Controlador para exponer servicios Web de órdenes de reposición.}, captionpos=b, label={lst:data-controller}]
@RestController
public class DataController{
	@Autowired
	private IOrdernesDao ordenesDao;
}
\end{lstlisting}

	Dentro de la clase declara como un bean que exponen servicios web se declaran métodos que serán expuestos como un servicio web, en el Código \ref{lst:data-controller} se muestra el servicio web para obtener una orden de reposición:
	\begin{enumerate}
		\item Línea 1: la anotación \textbf{RequestMapping} indica como se debe asociar el método URLs por medio de sus parámetros:
		\begin{enumerate}
			\item value: URL con la cuál es asociada el método, el parámetro entre corchetes indica que es variable.
			\item method: método de HTTP al cuál es asociado el método. 
		\end{enumerate}
		\item Línea 3: la anotación \textbf{PathVariable} indica que el valor del parámetro es tomado de la URL, en este caso se refiere al número identificador de la orden de reposición buscada.
		\item Línea 5: obtención de la orden de reposición.
	\end{enumerate}

\begin{lstlisting}[language=Java, caption={Servicio Web para obtener una orden de reposición.}, captionpos=b, label={lst:data-controller}]
@RequestMapping(value = "/_data_/orden/{id}",
				method = RequestMethod.GET)
public OrdenImss getOrden(@PathVariable("id") Long id) throws SQLException{

	return ordenesDao.getOrdenById(id);

}
\end{lstlisting}

	\item \textbf{ReportController}: expone servicios referentes a la generación de reportes, en el Código \ref{lst:report-controller} se muestra la estructura principal de esta clase:
\begin{lstlisting}[language=Java, caption={Controlador para exponer servicios Web de generación de reportes.}, captionpos=b, label={lst:report-controller}]
@Controller
public class ReportController{
	@Autowired
	private IOrdenesDao ordenesDao;
	
	@Autowired
	private IReportService reportService;
}
\end{lstlisting}

	En el Código \ref{lst:report-controller-gen} se muestra el servicio web para la generación de reportes:
	\begin{enumerate}
		\item Línea 1: la anotación \textbf{RequestMapping} indica como se debe asociar el método URLs por medio de sus parámetros:
		\begin{enumerate}
			\item value: URL con la cuál es asociada el método.
			\item method: método de HTTP al cuál es asociado el método.
			\item produces: indica el formato de respuesta, en este caso es un flujo de datos.
		\end{enumerate}
		\item Líneas 12 y 13: traducción de las fechas que acotan el reporte a un objeto Date.
		\item Línea 15: delegación de la generación del reporte al servicio de generación de reportes.
		\item Línea 16: si el reporte no es vacío, entonces se manda el reporte como flujo de bytes.
		\item Línea 17: si el reporte es vacío se manda un mensaje de error.
	\end{enumerate}

\begin{lstlisting}[language=Java, caption={Servicio Web para generar un reporte.}, captionpos=b, label={lst:report-controller-gen}]
@RequestMapping(value = "/_report_/generate",
				method = RequestMethod.GET,
				produces = "application/octet-stream")
public void generateReport(HttpServletRequest request,
						   HttpServletResponse response,
						   @RequestParam("reporte") ReportType rType,
						   @RequestParam("fecIni") String fecIni,
						   @RequestParam("fecFin") String fecFin,
						   @RequestParam("horIni") String horIni,
						   @RequestParam("horFin") String horFin)
					throws IOException{
	Date low = parseDate(fecIni, horIni);
	Date high = parseDate(fecFin, horFin);
	
	String pathfile = reportService.generate(rType, low, high);
	if(pathfile !=null && !pathfile.isEmpty()){
		writeOut(pathfile, request, response);
	}else{
		Writer out = response.getWriter();
		out.append("No se han encontrado resultados");
		out.flush();
	}
}
\end{lstlisting}

\end{enumerate}


\subsubsection{Implementación del Fronend}
El Frontend es conocido desde el servidor de aplicaciones como contenido estático, se compone de elementos estructurales y elementos de comportamiento.
\begin{itemize}
	\item Estructurales: documentos HTML, hojas de estilo e imágenes, definen como se muestra la información al usuario.
	\item Comportamiento: rutinas de Javascript, estas rutinas pueden modificar la estructura del HTML durante la ejecución, pero tales cambios no son persistentes, también se encargan de realizar peticiones al servidores de autorizaciones y al servidor de recursos.
\end{itemize}

La implementación del Frontend está basada en AngularJS, 


Los servicios para la gestión de órdenes de reposición y reportes.

\paragraph{Autenticación de usuario\\}

\paragraph{Consumo de servicios web\\}

\paragraph{Contenido estático\\}
\begin{enumerate}
	\item Páginas HTML
	\item Rutinas de Javascript
\end{enumerate}
