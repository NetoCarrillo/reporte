\subsection{Sistema de Archivos}
El componente Sistema de Archivos es el encargado de actuar como medio de comunicación entre el sistema AutoSA y el sistema de archivos del sistema operativo donde se ejecuta el sistema AutoSA. En particular este componente cumple con dos funciones:
\begin{enumerate}
	\item Lectura de configuración: leer un archivo de propiedades en formato llave valor.
	\item Escritura de archivos: escribir un flujo de bytes al sistema de archivos del sistema operativo
\end{enumerate}
Para las funciones anteriores se ha utilizado las bibliotecas de \textit{Java} para la lectura y escritura de archivos, en los siguientes apartados se mostrará la implementación de las funciones descritas anteriormente.

\subsubsection{Lectura de configuración}
La lectura de valores de configuración se hace mediante la lectura del archivo de propiedades dentro de un objeto de la clase \texttt{Properties} del paquete \texttt{java.util}, la lectura de la información del archivo de propiedades se logra utilizando el paquete \textit{Java IO}. A continuación se detalla la descripción anterior explicando el Código \ref{lst:fs-read}:

\begin{enumerate}
 	\item Creación de la instancia de la clase \texttt{Properties} (línea 1).
 	\item Apertura del flujo de datos del sistema de archivos (línea 2).
 	\item Lectura de los datos dentro del objeto \texttt{Properties} (línea 3).
 \end{enumerate}

\begin{lstlisting}[language=Java, caption={Lectura de un archivo de propiedades.}, captionpos=b, label={lst:fs-read}]
Properties props = new Properties();
try(InputStream is = new FileInputStream(new File(CONFIG_PATH, CONFIG_FILENAME));){
	props.load(is);
}
\end{lstlisting}

\subsubsection{Almacenamiento}
El almacenamiento de archivos en el proyecto AutoSA se utiliza cuando se desea generar un reporte, para esto es necesario contar con la plantilla del reporte y copiarla a un archivo nuevo con el fin de que posteriormente sea utilizado para vaciar el contenido del reporte.\\
Lo anterior se ejemplifica con los siguientes códigos:
\begin{enumerate}
	\item Obtiene la ruta de las plantillas en el sistema operativo como se muestra en el Código \ref{lst:fs-read-prop}.
	\begin{lstlisting}[language=Java, caption={Obtención de las rutas de las plantillas.}, captionpos=b, label={lst:fs-read-prop}]
	Path headersTmpl = Paths.get(props.getProperty("is.report.headers.tmpl"));
	Path contentTmpl = Paths.get(props.getProperty("is.report.content.tmpl"));
	Path headersPath = Paths.get(props.getProperty("is.report.out.path"), headersTmpl.getFileName().toString());
	Path contentPath = Paths.get(props.getProperty("is.report.out.path"), contentTmpl.getFileName().toString());
	\end{lstlisting}

	\item Utilizando la clases de \textit{Java} dedicadas al manejo de lectura y escritura se copia la plantilla, tal como se aprecia en el Código \ref{lst:fs-copy}.
	\begin{lstlisting}[language=Java, caption={Copia de archivos.}, captionpos=b, label={lst:fs-copy}]
	//Copiar plantillas
	try(FileOutputStream hfos = new FileOutputStream(headersPath.toFile(), false);
		FileOutputStream lfos = new FileOutputStream(contentPath.toFile(), false)){
		
		Files.copy(headersTmpl, hfos);
		Files.copy(contentTmpl, lfos);
	}
	\end{lstlisting}
\end{enumerate}
