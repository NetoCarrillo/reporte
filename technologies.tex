\chapter{Tecnologías utilizadas en la implementación}

\paragraph{Lenguajes de programación}
	\begin{itemize}
		\item Java
		\item Javascript
		\item Sahiscript
	\end{itemize}
\paragraph{Base de datos}
	\textcolor{red}{hacer nota: para respetar acuerdos de confidencialidad con el cliente no se mencionará de forma específica el sistema manejador de base de datos utilizado}
	\begin{itemize}
		\item SQL
		\item Estándares
	\end{itemize}

\paragraph{Herramienta de automatización}
	\begin{itemize}
		\item Sahi\footnote{Esta herramienta fue utilizada para la automatización con SA a petición de la farmacéutica.}
	\end{itemize}
\paragraph{Spring}
	\begin{itemize}
		\item Spring core\cite{SpringInAction}
		\item Spring data\cite{SpringInAction}
		\item Spring boot\cite{SpringBootInAction}
		\item Spring MVC\cite{SpringInAction}
		\item Spring REST\cite{SpringInAction}
		\item Spring security\cite{ProSpringSecurity}
	\end{itemize}

\paragraph{Mybatis}\cite{MyBatis}
En la ``listing'' \ref{lst:mybatis-interface} se muestra la interfaz de Java y el ``listing'' \ref{lst:mybatis-mapping} se muestra el DML. 
\begin{lstlisting}[language=Java, caption={Ejemplo de MyBatis}, label={lst:mybatis-interface}]
public interface IDomainUser{
	UsuarioDomain getUsuario(String name);
}
\end{lstlisting}

\begin{lstlisting}[language=XML, caption={Ejemplo de MyBatis}, label={lst:mybatis-mapping}]
<?xml version="1.0" encoding="UTF-8"?>
<!DOCTYPE mapper PUBLIC "-//mybatis.org//DTD Mapper 3.0//EN"
		"http://mybatis.org/dtd/mybatis-3-mapper.dtd">

<mapper namespace="IDomainUser">

	<resultMap id="rol" autoMapping="true"
			   type="RolDomain">
		<id property="rol" column="rol"/>
	</resultMap>

	<resultMap id="usuario" autoMapping="true"
			   type="UsuarioDomain">
		<id property="usuario" column="usuario"/>
		<collection property="roles"
					resultMap="rol"
					javaType="ArrayList"/>
	</resultMap>
	
	<!-- UsuarioDomain getUsuario(final String name); -->
	<select id="getUsuario" resultMap="usuario" useCache="false">
		SELECT u.*, r.*
		  FROM ralca.usuarios_domain u,
		       ralca.roles_domain r,
		       ralca.usuarios_roles ur
		 WHERE u.usuario = ur.usuario
		   AND r.rol = ur.rol
		   AND u.usuario = #{0};
	</select>
</mapper>
\end{lstlisting}



\paragraph{Velocity}
En el ``listing'' \ref{lst:velocity-template} se muestra un ejemplo de plantilla con Velocity.
\begin{lstlisting}[language=HTML, caption={Ejemplo de plantilla de Velocity}, label={lst:velocity-template}]
<tr>
	<td width="160"><b>No. de Orden: </b></td>
	<td>${numorden}</td>
</tr>
\end{lstlisting}

\paragraph{POI}

\paragraph{iTextPdf}\cite{iTextInAction}

\paragraph{Maven}\cite{MasteringApacheMaven3}
\begin{lstlisting}[language=XML, caption={Proyecto padre de Maven}, label={lst:maven-parent-project}]
<project xmlns="http://maven.apache.org/POM/4.0.0" xmlns:xsi="http://www.w3.org/2001/XMLSchema-instance"
	xsi:schemaLocation="http://maven.apache.org/POM/4.0.0 http://maven.apache.org/xsd/maven-4.0.0.xsd">
	<modelVersion>4.0.0</modelVersion>
	
	<groupId>com.meve</groupId>
	<artifactId>surtimiento-project</artifactId>
	<version>1</version>

	<modules>
		<module>surtimiento</module>
		<module>surtimiento-domain</module>
		<module>surtimiento-basic</module>
		<module>surtimiento-api</module>
		<module>surtimiento-dao-impl</module>
		<module>surtimiento-dao-jpa</module>
		<module>surtimiento-service-impl</module>
		<module>surtimiento-excel-plugin</module>
		<module>surtimiento-reporter</module>
		<module>surtimiento-worker</module>
	</modules>
</project>
\end{lstlisting}
