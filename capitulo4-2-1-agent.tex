\subsection{Agente}\label{sec:agente}
La implementación del componente \textbf{Agente} está escrito en rutinas de \textit{Sahi} (sección \ref{sec:sahi}). Primero se exponen los puntos relevantes en la implementación de las rutinas de respuesta y envío de órdenes de reposición para después señalar como se realiza la ejecución mediante la herramienta gráfica de \textit{Sahi}.

\subsubsection{Rutina para automatizar la respuesta de órdenes de reposición}\label{sec:aut-contestar}
La rutina para automatizar la respuesta de órdenes de reposición refleja el caso de uso \textbf{CU-CONTESTAR} que se describe en la sección \ref{cu-contestar}. A continuación se muestran las secciones de código más relevantes de la rutina que realiza la ejecución del caso de uso citado así como los subsecuentes (Figura \ref{fig:dia-casos-uso}). Los puntos relevantes en la implementación de la rutina son:
\begin{enumerate}
	\item Ingresar al \textit{Sistema de Abastecimiento}, en el Código \ref{lst:sah-session} se muestra la rutina para iniciar sesión en el \textit{Sistema de Abastecimiento}:
	\begin{enumerate}
		\item Las líneas 1 y 2 se llenan los campos de usuario y contraseña.
		\item La línea 3 se envía el formulario.
		\item La línea 4 redirige a la pantalla con el listado de órdenes de reposición.  
	\end{enumerate}

% espaciado
\pagebreak
% espaciado
	\begin{adjustwidth}{\listingfixwidth}{0pt}
	\begin{lstlisting}[language=Javascript, caption={Inicio de sesión en el \textit{Sistema de Abastecimiento}.}, captionpos=b, label={lst:sah-session}]
_setValue(_textbox("Usuario[1]"), $user);
_setValue(_password("Contras[1]"), $pwd);
_click(_submit("Ingresar al Sistema"));
_click(_image("Normal[2]"));
	\end{lstlisting}
	\end{adjustwidth}

	\item Recolectar las órdenes de reposición listadas, el Código \ref{lst:sah-save-news} muestra un resumen de la automatización para guardar el listado de órdenes de reposición (caso de uso \textbf{CU-GUARDAR-NUEVA}):
	\begin{enumerate}
		\item La línea 1 muestra la declaración del ciclo para recorrer el listado de órdenes de reposición.
		\item Las líneas 2 a 4 muestran como se extrae el valor de los datos de una orden de reposición.
		\item Las líneas 5 a 7 muestran la obtención de las URL para contestar y enviar las órdenes de reposición.
		\item La línea 8 realiza el almacenamiento de la nueva orden de reposición. 
	\end{enumerate}
	\begin{adjustwidth}{\listingfixwidth}{0pt}
	\begin{lstlisting}[language=Javascript, caption={Guardar lista de órdenes de reposición.}, captionpos=b, label={lst:sah-save-news}]
for(var $i = 1 + $errores; $i <= $rowCount; $i++){
	var $contrato = _getText(_table(1).rows[$i].cells[0]);
	var $solicitud = _getText(_table(1).rows[$i].cells[1]);
	var $numorden = _getText(_table(1).rows[$i].cells[2]);
	var $urlcon = "";
	_set($urlcon, _table(1).rows[$i].cells[6].childNodes[0].href);
	var $urlenv = $urlcon.replace("respoOra", "enviaOra");
	var $inserted = $persistence.insertOrden($contrato, $solicitud, $numorden, $expedicion, $almacen, $urlcon, $urlenv, $idSesion);
}
	\end{lstlisting}
	\end{adjustwidth}

	\item Contestar cada orden de reposición, el Código \ref{lst:sah-respond} muestra un resumen de la automatización para contestar una orden de reposición (caso de uso \textbf{CU-RESPONDER-ORDEN}):
	\begin{enumerate}
		\item Las líneas 1 y 2 muestran como se redirige al explorador a la URL para responder la orden de reposición.
		\item  La línea 3 muestra la asignación de un valor en el formulario para contestar la orden de reposición.
		\item La línea 4 realiza el envío del formulario.
		\item La línea 5 manda la actualización de la orden de reposición en la base de datos.
	\end{enumerate}
	\begin{adjustwidth}{\listingfixwidth}{0pt}
	\begin{lstlisting}[language=Javascript, caption={Responder orden de reposición.}, captionpos=b, label={lst:sah-respond}]
var $url = $entidad.getUrlCon();
_navigateTo($url);
_setValue(_textbox("Lote"), "SL");
_click(_submit("Agregar Captura"));
$logica.updateCantidad($contrato, $numorden, $cantidad, $idSesion);
	\end{lstlisting}
	\end{adjustwidth}

	\item Enviar las órdenes de reposición contestadas. El Código \ref{lst:sah-send} muestra un resumen de la automatización para enviar una orden de reposición (caso de uso \textbf{CU-ENVIAR-ORDEN}):
	\begin{enumerate}
		\item Las líneas 1 y 2 muestran como se redirige al explorador a la URL para enviar la orden de reposición.
		\item La líneas 3 a 6 crean un mapa con los datos de la orden de reposición.
		\item La línea 7 actualiza la orden de reposición en la base de datos. 
	\end{enumerate}
	\begin{adjustwidth}{\listingfixwidth}{0pt}
	\begin{lstlisting}[language=Javascript, caption={Enviar orden de reposición.}, captionpos=b, label={lst:sah-send}]
var $url = $entidad.getUrlEnv();
_navigateTo($url);
$mapa = new java.util.LinkedHashMap();
for(var $llave in $orden){
	$mapa.put($llave, $orden[$llave]);
}
$logica.updateOrden($mapa, $idSesion);
	\end{lstlisting}
	\end{adjustwidth}
\end{enumerate}

\subsubsection{Rutina para automatizar la verificación de órdenes de reposición canceladas}
La rutina para automatizar la verificación de órdenes de reposición canceladas refleja el caso de uso \textbf{CU-VERIFICAR} que se describe en la sección \ref{cu-verificar}, a continuación se muestran las secciones de código más relevantes de la rutina que realizan la ejecución del caso de uso citado así como los subsecuentes (Figura \ref{fig:dia-casos-uso}). Los puntos relevantes en la implementación de la rutina son los siguientes:
\begin{enumerate}
	\item Ingresar al \textit{Sistema de Abastecimiento}, el ingreso al sistema es idéntico a la rutina mostrada en la sección anterior (sección \ref{sec:aut-contestar}).

	\item El Código \ref{lst:sah-search} muestra un resumen de la automatización de los criterios de búsqueda:
	\begin{enumerate}
		\item La línea 1 muestra la consulta de las fechas que acotan la búsqueda.
		\item Las líneas 2 a 5 realizan la búsqueda en el \textit{Sistema de Abastecimiento}.
		\item Las líneas 6 y 7 extraen la información de las órdenes de reposición que muestra el \textit{Sistema de Abastecimiento} como resultado de la búsqueda.
		\item Las líneas 8 y 9 utilizan el componente \textbf{Lógica de Automatización} para actualizar en la base de datos el estado de las órdenes de reposición canceladas.
	\end{enumerate}
	\begin{adjustwidth}{\listingfixwidth}{0pt}
	\begin{lstlisting}[language=Javascript, caption={Responder orden de reposición.}, captionpos=b, label={lst:sah-search}]
$dateLimits = $logica.getDateLimits('dd/MM/yyyy');
_setSelected(_select("OrdStt"), "Canceladas");
_setValue(_textbox("OrdFeD"), $dateLimits[0]);
_setValue(_textbox("OrdFeA"), $dateLimits[1]);
_click(_submit(0));
var $html = '';
_set($html, _table(5).innerHTML);
$estado = $logica.getCancelStatus();
$logica.setSaiStatus($html, $estado, $idSesion);
	\end{lstlisting}
	\end{adjustwidth}
\end{enumerate}
