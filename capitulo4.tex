\chapter{Implementación}\label{cap4}


\section{Implementación de base de datos}
\textcolor{blue}{Modelo relacional}\\
\subsection{SQL}
\textcolor{blue}{ANSII y cosas de esas}
\subsection{Tablas de órdenes}
\subsection{Catálogos}
\subsection{Tablas de usuarios}

\section{Implmenetación de genración de reportes}

\textcolor{blue}{En el diseño debe agregarse un diagrama del funcionamiento de reportes: seleccionar templates... agregar patrones de diseño}\\
\subsection{Motor de plantillas}
\textcolor{blue}{aquí platico de volicity y como funciona el motor}\\
\textcolor{blue}{aquí va la configuración}\\
\textcolor{blue}{tipos de reporte}

\section{Implementación de automatización para SA}
\subsection{Bibliotecas para las rutinas de automatización}
\subsection{Rutinas de automatización}

\section{Implementación de interfaz web}
\textcolor{blue}{Las secciones de esta parte serán presentadas por capas desde datos hasta la vista}\\
\textcolor{blue}{Presentar el patron MVC}\\
\subsection{Acceso}
\textcolor{blue}{Spring security}\\
\textcolor{blue}{Cifrado de contraseña}\\
\subsection{Generación de reportes}
\subsection{Búsqueda}

\subsection{Administración de catálogos}

\subsection{Visualización de órden de reposición}

\subsection{Edición de órden de reposición}
