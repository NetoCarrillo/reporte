\chapter{Implementación}\label{cap4}

\section{Aspectos generales del la implementación}
\textcolor{red}{Esta sección contiene las tecnologías más relevantes para la implmentación del sistema}
\paragraph{Lenguajes de programación}
	\begin{itemize}
		\item Java
		\item Javascript
		\item Sahiscript
	\end{itemize}
\paragraph{Base de datos}
	\textcolor{red}{hacer nota: para respetar acuerdos de confidencialidad con el cliente no se mencionará de forma específica el sistema manejador de base de datos utilizado}
	\begin{itemize}
		\item SQL
		\item Estándares
	\end{itemize}

\paragraph{Herramienta de automatización}
	\begin{itemize}
		\item Sahi\footnote{Esta herramienta fue utilizada para la automatización con SA a petición de la farmacéutica.}
	\end{itemize}
\paragraph{Spring}
	\begin{itemize}
		\item Spring core\cite{SpringInAction}
		\item Spring data\cite{SpringInAction}
		\item Spring boot\cite{SpringBootInAction}
		\item Spring MVC\cite{SpringInAction}
		\item Spring REST\cite{SpringInAction}
		\item Spring security\cite{ProSpringSecurity}
	\end{itemize}

\paragraph{Mybatis}\cite{MyBatis}
En la ``listing'' \ref{lst:mybatis-interface} se muestra la interfaz de Java y el ``listing'' \ref{lst:mybatis-mapping} se muestra el DML. 
\begin{lstlisting}[language=Java, caption={Ejemplo de MyBatis}, label={lst:mybatis-interface}]
public interface IDomainUser{
	UsuarioDomain getUsuario(String name);
}
\end{lstlisting}

\begin{lstlisting}[language=XML, caption={Ejemplo de MyBatis}, label={lst:mybatis-mapping}]
<?xml version="1.0" encoding="UTF-8"?>
<!DOCTYPE mapper PUBLIC "-//mybatis.org//DTD Mapper 3.0//EN"
		"http://mybatis.org/dtd/mybatis-3-mapper.dtd">

<mapper namespace="IDomainUser">

	<resultMap id="rol" autoMapping="true"
			   type="RolDomain">
		<id property="rol" column="rol"/>
	</resultMap>

	<resultMap id="usuario" autoMapping="true"
			   type="UsuarioDomain">
		<id property="usuario" column="usuario"/>
		<collection property="roles"
					resultMap="rol"
					javaType="ArrayList"/>
	</resultMap>
	
	<!-- UsuarioDomain getUsuario(final String name); -->
	<select id="getUsuario" resultMap="usuario" useCache="false">
		SELECT u.*, r.*
		  FROM ralca.usuarios_domain u,
		       ralca.roles_domain r,
		       ralca.usuarios_roles ur
		 WHERE u.usuario = ur.usuario
		   AND r.rol = ur.rol
		   AND u.usuario = #{0};
	</select>
</mapper>
\end{lstlisting}



\paragraph{Velocity}
En el ``listing'' \ref{lst:velocity-template} se muestra un ejemplo de plantilla con Velocity.
\begin{lstlisting}[language=HTML, caption={Ejemplo de plantilla de Velocity}, label={lst:velocity-template}]
<tr>
	<td width="160"><b>No. de Orden: </b></td>
	<td>${numorden}</td>
</tr>
\end{lstlisting}

\paragraph{POI}

\paragraph{iTextPdf}\cite{iTextInAction}

\paragraph{Maven}\cite{MasteringApacheMaven3}
\begin{lstlisting}[language=XML, caption={Proyecto padre de Maven}, label={lst:maven-parent-project}]
<project xmlns="http://maven.apache.org/POM/4.0.0" xmlns:xsi="http://www.w3.org/2001/XMLSchema-instance"
	xsi:schemaLocation="http://maven.apache.org/POM/4.0.0 http://maven.apache.org/xsd/maven-4.0.0.xsd">
	<modelVersion>4.0.0</modelVersion>
	
	<groupId>com.meve</groupId>
	<artifactId>surtimiento-project</artifactId>
	<version>1</version>

	<modules>
		<module>surtimiento</module>
		<module>surtimiento-domain</module>
		<module>surtimiento-basic</module>
		<module>surtimiento-api</module>
		<module>surtimiento-dao-impl</module>
		<module>surtimiento-dao-jpa</module>
		<module>surtimiento-service-impl</module>
		<module>surtimiento-excel-plugin</module>
		<module>surtimiento-reporter</module>
		<module>surtimiento-worker</module>
	</modules>
</project>
\end{lstlisting}


\begin{lstlisting}[language=XML, caption={Configuración de la aplicación web}, label={lst:maven-spring-boot-webapp}]
<project>
	...
	<artifactId>surtimiento-webapp</artifactId>

	<parent>
		<groupId>org.springframework.boot</groupId>
		<artifactId>spring-boot-starter-parent</artifactId>
	</parent>
	
	<dependencies>
		<dependency>
			<groupId>org.springframework.boot</groupId>
			<artifactId>spring-boot-starter-web</artifactId>
		</dependency>
		<dependency>
			<groupId>org.springframework.boot</groupId>
			<artifactId>spring-boot-starter-security</artifactId>
		</dependency>
	
		<!-- Domain -->
		<dependency>
			<groupId>com.meve</groupId>
			<artifactId>surtimiento-domain</artifactId>
		</dependency>
		<dependency>
			<groupId>com.meve</groupId>
			<artifactId>surtimiento-api</artifactId>
		</dependency>
		<dependency>
			<groupId>com.meve</groupId>
			<artifactId>surtimiento-dao-jpa</artifactId>
		</dependency>
		<dependency>
			<groupId>com.meve</groupId>
			<artifactId>surtimiento-service-impl</artifactId>
		</dependency>
		<dependency>
			<groupId>com.meve</groupId>
			<artifactId>surtimiento-reporter</artifactId>
		</dependency>
		<!-- Domain -->
		
		<!-- Persistence -->
		<dependency>
			<groupId>commons-dbcp</groupId>
			<artifactId>commons-dbcp</artifactId>
		</dependency>
	    <dependency>
	    	<groupId>javax.persistence</groupId>
	    	<artifactId>persistence-api</artifactId>
	    	<version>1.0</version>
	    </dependency>
		<dependency>
			<groupId>mysql</groupId>
			<artifactId>mysql-connector-java</artifactId>
		</dependency>
		<!-- Persistence -->
	</dependencies>

	<build>
		<filters>
			<filter>profiles/${build.profile.id}/config.properties</filter>
		</filters>
		
		<resources>
			<resource>
				<filtering>true</filtering>
				<directory>${project.basedir}/src/main/resources</directory>
			</resource>
			<resource>
				<directory>${project.build.directory}/generated-resources</directory>
			</resource>
		</resources>
		
		<plugins>
			<plugin>
				<groupId>org.springframework.boot</groupId>
				<artifactId>spring-boot-maven-plugin</artifactId>
				<configuration>
					<finalName>${jar.file.name}</finalName>
				</configuration>
			</plugin>
			<plugin>
				<artifactId>maven-resources-plugin</artifactId>
				<executions>
					<execution>
						<!-- Serves *only* to filter the wro.xml so it can get an absolute path 
							for the project -->
						<id>copy-resources</id>
						<phase>validate</phase>
						<goals>
							<goal>copy-resources</goal>
						</goals>
						<configuration>
							<outputDirectory>${basedir}/target/wro</outputDirectory>
							<resources>
								<resource>
									<directory>src/main/wro</directory>
									<filtering>true</filtering>
								</resource>
							</resources>
							<delimiters>
								<delimiter>@</delimiter>
							</delimiters>
							<useDefaultDelimiters>false</useDefaultDelimiters>
						</configuration>
					</execution>
				</executions>
			</plugin>
			<plugin>
				<groupId>ro.isdc.wro4j</groupId>
				<artifactId>wro4j-maven-plugin</artifactId>
				<version>1.7.6</version>
				<executions>
					<execution>
						<phase>generate-resources</phase>
						<goals>
							<goal>run</goal>
						</goals>
					</execution>
				</executions>
				<configuration>
					<wroManagerFactory>ro.isdc.wro.maven.plugin.manager.factory.ConfigurableWroManagerFactory</wroManagerFactory>
					<cssDestinationFolder>${project.build.directory}/generated-resources/static/css</cssDestinationFolder>
					<jsDestinationFolder>${project.build.directory}/generated-resources/static/js</jsDestinationFolder>
					<wroFile>${project.build.directory}/wro/wro.xml</wroFile>
					<extraConfigFile>${basedir}/src/main/wro/wro.properties</extraConfigFile>
					<contextFolder>${basedir}/src/main/wro</contextFolder>
				</configuration>
				<dependencies>
					<dependency>
						<groupId>org.webjars</groupId>
						<artifactId>jquery</artifactId>
						<version>2.1.1</version>
					</dependency>
					<dependency>
						<groupId>org.webjars</groupId>
						<artifactId>angularjs</artifactId>
						<version>1.3.8</version>
					</dependency>
					<dependency>
						<groupId>org.webjars</groupId>
						<artifactId>bootstrap</artifactId>
						<version>3.2.0</version>
					</dependency>
				</dependencies>
			</plugin>
		</plugins>
	</build>
	
	<profiles>
		<!-- Developing -->
		<profile>
			<id>dev</id>
			<activation>
				<activeByDefault>true</activeByDefault>
			</activation>
			<properties>
				<build.profile.id>dev</build.profile.id>
				<jar.file.name>webapp_${build.profile.id}</jar.file.name>
			</properties>
		</profile>
		
		<!-- Testing -->
		<profile>
			<id>test</id>
			<properties>
				<build.profile.id>test</build.profile.id>
				<jar.file.name>webapp_${build.profile.id}</jar.file.name>
			</properties>
		</profile>
		
		<!-- Production -->
		<profile>
			<id>prod</id>
			<properties>
				<build.profile.id>prod</build.profile.id>
				<jar.file.name>webapp</jar.file.name>
			</properties>
		</profile>
	</profiles>
</project>	
\end{lstlisting}


\section{Implementación de base de datos}
\textcolor{blue}{La base de datos utilizada para el proyecto es una base de datos relacional compatible con el modelo ANSI}\\
Mencionar: 
\begin{itemize}
	\item Concepto de índice.
	\item Índices para búsquedas de órdenes.
	\item Índices para reportes.
	\item Concepto de vista.
	\item Vistas para reportes.
	\item Concepto de transacción.
\end{itemize}


\section{Implementación de los componentes}

\subsection{Persistencia}
\subsubsection{Tecnologías utilizadas}
\begin{enumerate}
	\item SQL 
		\begin{enumerate}
			\item DDL: CREATE TABLE, VIEW
			\item DML: INSERT, UPDATE, DELETE, SELECT
		\end{enumerate}
	\item JDBC 			(Patrón factory)
	\item JPA			(Patrón proxy)
	\item SpringData	(Patrón proxy)
\end{enumerate}

\textcolor{blue}{
	Acá un diagrama que muestra las partes importantes del componente
}

\subsection{Sistema de archivos}
	\paragraph{Configuración\\}
		\textbf{obtener-propiedad}
	\paragraph{Almacenamiento\\}
		\textbf{guardar-archivo}
\subsection{Generador de reportes}
	\paragraph{Acuse\\}
		\textbf{generar-acuse-envio}
	\paragraph{Generación\\}
 		\textbf{generar-reporte-ordenes}
\subsection{Lógica de automatización}
	\paragraph{Respuesta\\}
		\textbf{guardar-orden-nueva}\\
		\textbf{obtener-datos-respuesta}\\
		\textbf{actualizar-orden-contestada}\\
		\textbf{guardar-orden-enviada}\\
		\textbf{obtener-acuse-envio}
	\paragraph{Verificación\\}
		\textbf{obtener-rango-fechas-verificar}\\
		\textbf{actualizar-estado-sa}


\section{Implmenetación de genración de reportes}

\textcolor{blue}{En el diseño debe agregarse un diagrama del funcionamiento de reportes: seleccionar templates... agregar patrones de diseño}\\
\subsection{Motor de plantillas}
\textcolor{blue}{aquí platico de volicity y como funciona el motor}\\
\textcolor{blue}{aquí va la configuración}\\
\textcolor{blue}{tipos de reporte}

\section{Implementación de automatización para SA}
\subsection{Bibliotecas para las rutinas de automatización}
\subsection{Rutinas de automatización}

\section{Implementación de interfaz web}
\textcolor{blue}{Las secciones de esta parte serán presentadas por capas desde datos hasta la vista}\\
\textcolor{blue}{Presentar el patron MVC}\\
\subsection{Acceso}
\textcolor{blue}{Spring security}\\
\textcolor{blue}{Cifrado de contraseña}\\
\subsection{Generación de reportes}
\subsection{Búsqueda}

\subsection{Administración de catálogos}

\subsection{Visualización de órden de reposición}

\subsection{Edición de órden de reposición}
