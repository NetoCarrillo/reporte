\subsection{Lógica de automatización}
El componente de lógica de automatización provee de la información necesaria a las rutinas automatizadas. Esta información puede proceder de diferentes fuentes y a cada fuente corresponde un componente diferente:
\begin{enumerate}
 	\item Si la fuente es un archivo en el sistema de archivos del sistema operativo, se obtiene la información utilizando el componente \textbf{Sistema de Archivos}.
 	\item Si la fuente es una base de datos, se tiene acceso miente el componente \textbf{Persistencia}.
 	\item Si la fuente son las reglas de negocio, se tiene acceso mediante el componente \textbf{Lógica de automatización} que es el encargado de aplicar los cálculos correspondientes a la lógica de negocio.
\end{enumerate}
La implementación de este componente ha sido escrita en el lenguaje de programación \textit{Java}. A continuación se muestran los puntos relevantes de la implementación de las interfaces mencionadas.
\paragraph{\indent Interfaz Respuesta}
\begin{enumerate}
	\item guardar-orden-nueva: la implementación de esta operación (Código \ref{lst:la-save-new}) muestra como se delega la aplicación del almacenamiento de una nueva orden de reposición al componente \textbf{Persistencia}.
	\begin{lstlisting}[language=Java, caption={Delegación del almacenamiento de una nueva orden de reposición.}, captionpos=b, label={lst:la-save-new}]
Orden orden = new Orden();
orden.setContrato(contrato);
orden.setOrden(ordenLong);
orden.setFechaExpedicion(fechaExpedicion);
orden.setUrlCon(urlCon);
orden.setUrlEnv(urlEnv);
orden.setEstatus(1);
orden.setIdSesionEstatus(idSesion);

ordenesDAO.insertOrden(orden);
	\end{lstlisting}

	\item obtener-datos-respuesta: la implementación del cálculo de fechas de fabricación y de caducidad descritas para el caso de uso \textbf{CU-RESPONDER-ORDEN}, el Código \ref{lst:la-dates} muestra el método en lenguaje \textit{Java}:
	\begin{enumerate}
		\item En las líneas 2 a 7 se calcula el año para las fechas de fabricación y de caducidad.
		\item En las líneas 9 a 11 se construyen las fechas de fabricación y de caducidad.
	\end{enumerate}
	\begin{lstlisting}[language=Java, caption={Método para calcular las fechas de fabricación y caducidad.}, captionpos=b, label={lst:la-dates}]
public String[] getFechasFab(){
	Calendar today = GregorianCalendar.getInstance();
	int anho;
	if(Calendar.DECEMBER == today.get(Calendar.MONTH)){
		anho = today.get(Calendar.YEAR) + 1;
	}else{
		anho = today.get(Calendar.YEAR);
	}
	String[] fechas = new String[2];
	fechas[0] = "01/01/" + anho;
	fechas[1] = "31/12/" + anho;
	
	return fechas;
}
	\end{lstlisting}

	\item obtener-acuse-envio: esta operación principalmente obtiene las rutas en el sistema de archivos para la generación del acuse de envío. Posteriormente delega la generación del acuse al componente \textbf{Generación de reportes}. En el Código \ref{lst:la-acuse} se muestra el código de \textit{Java} que realiza los pasos anteriores.
	\begin{enumerate}
		\item La línea 1 asigna el nombre del archivo como el número de la orden de reposición.
		\item La línea 2 obtiene la plantilla del acuse de envío para el \textit{Instituto de Salud}.
		\item La línea 3 obtiene la ruta donde se depositan los archivos auxiliares en la generación del acuse de envío.
		\item Las líneas 4 a 7 construyen la ruta de directorios donde se depositará el acuse de envío.
		\item La línea 8 utiliza el componente \textbf{Generador de Reportes} para la creación del acuse de envío.
	\end{enumerate}
	\begin{lstlisting}[language=Java, caption={Generación del acuse de envío.}, captionpos=b, label={lst:la-acuse}]
String filename = params.get("numorden");
String template = properties.getProperty("is.template.html");
File outhtmldir = new File(properties.getProperty("is.output.dir"), filename + ".html");
SimpleDateFormat sdf = new SimpleDateFormat("yyyy MMMM dd", new Locale("es", "MX"));
String[] date = sdf.format(new Date()).split(" ");
File outputdir = new File(properties.getProperty("is.output.pdf"), String.format(REPORT_DIR_TMPL, date[0], date[1], date[2]));
outputdir.mkdirs();
snapShotService.takeSnapShot(params, filename, template, outhtmldir, outputdir);
	\end{lstlisting}
\end{enumerate}

\paragraph{\indent Interfaz Verificación}
\begin{enumerate}
	\item obtener-rango-fechas-verificar: el Código \ref{lst:la-date-search} muestra el cálculo de las fechas necesarias para el formulario de búsqueda de órdenes de reposición.
	\begin{enumerate}
		\item La línea 5 muestra la obtención de la fecha mayor (día actual).
		\item La líneas 6 y 7 muestran la obtención de la fecha menor (60 días antes de la fecha actual).
	\end{enumerate}
	\begin{lstlisting}[language=Java, caption={Cálculo del rango de fechas para buscar órdenes de reposición canceladas.}, captionpos=b, label={lst:la-date-search}]
DateFormat dateFormat = new SimpleDateFormat(format);
Calendar cal = GregorianCalendar.getInstance();
String[] dates = new String[2];
dates[1] = dateFormat.format(cal.getTime());
cal.add(Calendar.DAY_OF_YEAR, -60);
dates[0] = dateFormat.format(cal.getTime());
	\end{lstlisting}

	\item actualizar-estado-sa: esta operación realiza la actualización de las órdenes de reposición canceladas, el Código \ref{lst:la-validate} muestra el código escrito en \textit{Java}.
	\begin{enumerate}
		\item La línea 1 obtiene las órdenes de reposición canceladas del listado de órdenes de reposición encontradas.
		\item La línea 2 actualiza el estado de las órdenes de reposición.
	\end{enumerate}
	\begin{lstlisting}[language=Java, caption={Actualización de órdenes de reposición canceladas.}, captionpos=b, label={lst:la-validate}]
	List<Orden> ordenes = xmlReader.getCancelled(htmlTable);
	persistence.updateSaiStatus(ordenes, status);
	\end{lstlisting}
\end{enumerate}
